%! Author = Андрей
%! Date = 05.11.2022

%----------------------------------------------------------------------------------------
%	TITLE AND CONTACT INFORMATION
%----------------------------------------------------------------------------------------

    \begin{minipage}[t]{0.45\textwidth} % 45% of the page width for name
        \vspace{-\baselineskip} % Required for vertically aligning minipages

        % If your name is very short, use just one of the lines below
        % If your name is very long, reduce the font size or make the minipage wider and reduce the others proportionately
        \colorbox{black}{{\HUGE\textcolor{white}{\textbf{\MakeUppercase{Andrei}}}}} % First name

        \colorbox{black}{{\HUGE\textcolor{white}{\textbf{\MakeUppercase{Briukhov}}}}} % Last name

        \vspace{6pt}

        {\huge Software developer} % Career or current job title
    \end{minipage}
    \begin{minipage}[t]{0.275\textwidth} % 27.5% of the page width for the first row of icons
        \vspace{-\baselineskip} % Required for vertically aligning minipages

        % The first parameter is the FontAwesome icon name, the second is the box size and the third is the text
        % Other icons can be found by referring to fontawesome.pdf (supplied with the template) and using the word after \fa in the command for the icon you want
        %\icon{MapMarker}{12}{}\\

        \icon{Globe}{12}{Brazil, São Paulo}\\
        \icon{Calendar}{12}{25-05-1986, 36 yo}\\
        \icon{FileText}{12}{Generated in \LaTeX}\\
        \icon{Phone}{12}{+55 (11) 97791-8776}\\
        \icon{At}{12}{\href{mailto:andreybr611@gmail.com}{andreybr611@gmail.com}}

    \end{minipage}
    \begin{minipage}[t]{0.275\textwidth} % 27.5% of the page width for the second row of icons
        \vspace{-\baselineskip} % Required for vertically aligning minipages

        % The first parameter is the FontAwesome icon name, the second is the box size and the third is the text
        % Other icons can be found by referring to fontawesome.pdf (supplied with the template) and using the word after \fa in the command for the icon you want
        \icon{Github}{12}{\href{https://github.com/tbw777}{tbw777}}\\
        \icon{Linkedin}{12}{\href{https://www.linkedin.com/in/andrey-bryukhov-ab5ba5254/}{andrey-bryukhov-ab5ba5254}} \\
        \icon{Skype}{12}{\href{https://join.skype.com/invite/o4lg3hvQNmjK}{invite/o4lg3hvQNmjK}} \\
        \icon{Twitter}{12}{\href{https://twitter.com/ABriukhov}{ABriukhov}} \\
        \icon{Whatsapp}{12}{\href{https://wa.me/79167372164}{WA click to chat}}
    \end{minipage}

    \vspace{0.5cm}

%----------------------------------------------------------------------------------------
%	INTRODUCTION, SKILLS AND TECHNOLOGIES
%----------------------------------------------------------------------------------------

    \cvsect{Who am I and why?}

    \begin{minipage}[t]{0.4\textwidth} % 40% of the page width for the introduction text
        \vspace{-\baselineskip} % Required for vertically aligning minipages

%	\lorem \lorem \lorem \lorem \lorem\\ % Dummy text%
        Comecei a escrever programas na escola e não consegui parar desde então. Na hora de escolher uma solução, prefiro versatilidade, consistência, sempre buscando soluções sistêmicas escaláveis.
    \end{minipage}
    \hfill % Whitespace between
    \begin{minipage}[t]{0.5\textwidth} % 50% of the page for the skills bar chart

        %\cvsect{Hobbies}

        \vspace{-\baselineskip} % Required for vertically aligning minipages
        \begin{barchart}{5.5}
            \baritem{Kotlin}{25}
            \baritem{Java}{100}
            \baritem{JavaScript}{40}
            \baritem{C/CPP}{50}
            \baritem{Assembler x86}{70}
        \end{barchart}
    \end{minipage}

%\begin{center}
%	\bubbles{5/Eclipse, 6/git, 4/Office, 3/Inkscape, 3/Blender}
%\end{center}

