%! Author = Андрей
%! Date = 05.11.2022

%----------------------------------------------------------------------------------------
%	EXPERIENCE
%----------------------------------------------------------------------------------------

\cvsect{EXPERIÊNCIA}

\begin{entrylist}
    \entry
    {09/2022\\\footnotesize{Solicitação única}}
    {Consultor}
    {\enquote{RSHB} banco}
    {
        Forneceu análises e recomendações para reestruturação de projetos e gerenciamento de tarefas
    }

    \entry
    {02/2021 -- 07/2022\\\footnotesize{1 ano 6 meses\\tempo total + finais de semana}}
    {Chefe de um grupo de trabalho}
    {Fazum em Moscow Centro de Inovação \enquote{Skolkovo}}
    {
        Construiu uma solução baseada em NetBeans RCP (não IDE) \\
    Criou novos processos e serviços Camunda com Spring (incluindo Swagger/OpenAPI/Feign) e Gradle \\
    Documentação universal autogerada criada com Git e LaTeX \\
    Criei um serviço de relatórios que gera dados para Power BI de diferentes fontes: Jira, Elastic, Kelycloak \\
    Disponibilização de consultas bancárias diárias na plataforma \\
    \texttt{Java 6-17}\slashsep\texttt{SQL}\slashsep\texttt{Camunda}\slashsep\texttt{Kotlin}\slashsep\texttt{Elastic}\slashsep\texttt{Linux}\slashsep\texttt{Docker}\slashsep\texttt{Swagger \& OpenAPI}
    }

    \entry
    {05/2018 -- 02/2021\\\footnotesize{2 anos 10 meses\\tempo total}}
    {Liderança da equipe}
    {MOCIKT}
    {
        Manutenção e desenvolvimento do portal de serviços \url{https://uslugi.mosreg.ru} \\
        Projetos transferidos de um fornecedor \\
        Projeto concluído e desenvolvimento de novos serviços \\
        Corrigidos erros do fornecedor na produção \\
        Foi feito treino de tarefas para os colegas \\
        Revisão de tarefa/código \\
        Interação com outros departamentos ao acoplar sistemas \\
        Ajude os colegas a resolver vários problemas \\
        Monitoramento de produtividade, hot fixes \\
    Migração de TeamCity para Gitlab CI \\
    Tamanho da equipe cerca de 8 trabalhadores: (back e front developers, devops, outros) \\
    \texttt{Java 4-15}\slashsep\texttt{Kotlin}\slashsep\texttt{Spring}\slashsep\texttt{Postgres}\slashsep\texttt{Elastic}\slashsep\texttt{Gitlab CI}\slashsep\texttt{Teamcity}\slashsep\texttt{Docker}
	}

    \entry
    {08/2017 -- 05/2018\\\footnotesize{10 meses\\tempo total + weekends}}
    {Java desenvolvedor}
    {PSB Bank}
    {
        Suporte e desenvolvimento de aplicações Java EE \\
        \texttt{Java SE 8 \& Java EE 6}\slashsep\texttt{Oracle SQL}\slashsep\texttt{IBM WebSphere}
    }
    
    \entry
    {07/2017 -- 08/2017\\\footnotesize{2 meses\\tempo total}}
    {Java desenvolvedor}
    {Paramitec}
    {
        Suporte e desenvolvimento de aplicativos do Ministério das Relações Exteriores. \\
        Saí depois de receber uma oferta do PS Bank. \\
        \texttt{Java 8}\slashsep\texttt{PostgreSQL}\slashsep\texttt{Tomcat}
    }
    
    \entry
    {01/2016 -- 06/2017\\\footnotesize{1 ano 6 meses\\tempo total}}
    {Java EE/Android/Node.js/PHP desenvolvedor}
    {Octopod}
    {
        Criou uma extensão em CUBA.platform \\
        Pesquisei o sistema de fluxo de trabalho de tese e docsvision para a possibilidade de expansão \\
        Desenvolvi um mini site usando um banco de dados gráfico \\
        Desenvolvi uma aplicação móvel para a empresa X5 \\
        Desenvolvi um chatbot para o microsoft digital forum 2016 \\
        Desenvolvi aplicativo corporativo JAX-RS (Java EE) para dispositivos móveis\\
    \texttt{Java 8}\slashsep\texttt{Websphere}\slashsep\texttt{Hibernate}\slashsep\texttt{Cuba}\slashsep\texttt{Android}\slashsep\texttt{SAP OData}\slashsep\texttt{Node.js}\slashsep\texttt{bots}
	}
    
    \entry
    {04/2013 -- 09/2015\\\footnotesize{2 anos 6 meses\\tempo total}}
    {Java desenvolvedor}
    {Supertel}
    {
        Suporte e manutenção de um sistema de software de telecomunicações em duas versões \\
        Ferramenta de migração de dados desenvolvida \\
        Forneceu análises e recomendações para um antigo projeto montador \\
        \texttt{Java 7}\slashsep\texttt{SNMP}\slashsep\texttt{Maven}\slashsep\texttt{JUnit}\slashsep\texttt{PostreSQL}\slashsep\texttt{Glassfish}\slashsep\texttt{NetBeans RCP}
    }

    \entry
    {07/2005 -- 03/2013\\\footnotesize{7 anos 9 meses\\tempo total}}
    {System software desenvolvedor}
    {Radioavionika}
    {
        \vspace{-4mm}
        \begin{itemize}[leftmargin=.12in]
            \item Muitos componentes desenvolvidos para sistema operacional de tempo real:
            \begin{itemize}
                \setlength\itemsep{0em}
                \item Implementação do modo protegido Intel x86
                \item Subsistema de controle de segurança multinível em tempo real
                \item Autotestes de execução multinível em tempo real
                \item 16 \& bootloaders de 32 bits com hacks de memória e usando: modos real/irreal/protegido
                \item drivers VESA
                \item Linker PE/DLL próprio para fornecer suporte de seção ROM e código separado para arquivos diferentes
            \end{itemize}
            \item Geradores de logs flexíveis
            \item Vários programas para logs de análise, memória e etapas de compilação
            \item Thin client automatizado para carga intelectual de dados em hardware sem monitor
            \item Vários programas experimentais para KolibriOS: controle de cache e medição de tempos de código
            \item Documentação de parte dos componentes
        \end{itemize}
    \texttt{FASM \& TASM}\slashsep\texttt{Bochs}\slashsep\texttt{Linux}\slashsep\texttt{GCC}\slashsep\texttt{Doxygen}\slashsep\texttt{C \& C++}\slashsep\texttt{Gimpel PC Lint}\slashsep\texttt{Java 6}}

\end{entrylist}
